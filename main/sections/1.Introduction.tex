\section{Introduction}
\subsection{Background}
Televised dance competitions combine performance and popularity under a weekly elimination system.
lemph{Dancing with the Stars} (DWTS) is a representative example, where outcomes depend on judges' scores and fan votes.
Judges' scores are observed and standardized, but fan vote totals are not disclosed and remain proprietary.
Thus, the elimination process is only partially observable: eliminations are known, while the underlying vote distribution is hidden.
This motivates mathematical modeling.
By reconstructing plausible vote shares consistent with observed outcomes, we can evaluate how different combination rules affect fairness, predictability, and controversy.
We also examine whether contestant attributes and professional partners influence success beyond
dancing performance.

\subsection{Literature Review}
Voting-based competitions have been examined through social choice theory and fairness analysis, especially in how rankings or normalized scores are aggregated into final outcomes. 
When important variables are not directly observed, Bayesian inference and simulation-based methods such as Approximate Bayesian Computation (ABC) provide practical tools to reconstruct latent preferences with uncertainty.

\subsection{The Description of the Problem}
We study DWTS eliminations across multiple seasons, where weekly outcomes depend on judges' scores and undisclosed fan votes. 
The dataset provides judge scores, contestant attributes, season/week identifiers, and the eliminated couple each week.

Our objectives are:
{\renewcommand{\baselinestretch}{1.0}\selectfont
\begin{enumerate}
    \item \textbf{Fan Vote Inference:} Estimate weekly fan vote shares consistent with observed eliminations and quantify uncertainty.
    \item \textbf{Voting Rule Comparison:} Compare rank-based and percentage-based systems, and test a bottom-two judge decision mechanism.
    \item \textbf{Feature Impact Analysis:} Evaluate how professional partners and celebrity characteristics relate to scores, popularity, and survival.
    \item \textbf{Rule Design:} Propose and justify a fairer voting procedure supported by quantitative evidence.
\end{enumerate}
}

\subsection{Our Work}
We propose a unified pipeline combining Bayesian reconstruction, rule-based simulation, and interpretable predictive modeling.

\textbf{Task 1: Fan vote reconstruction.}
We infer latent fan vote shares for each active couple using a Dirichlet prior and an ABC-style procedure that generates plausible vote vectors consistent with observed eliminations.

\textbf{Task 2: Rule comparison.}
Using inferred fan preferences, we replay historical weeks under multiple aggregation rules and summarize agreement rates, flip cases, and disagreement patterns.

\textbf{Task 3: Feature impact.}
We use regularized regression to separate performance-driven effects from popularity-driven effects and identify influential contestant and partner attributes.

\textbf{Task 4: New voting system proposal.}
Based on empirical findings, we propose and backtest an alternative voting scheme that aims to improve fairness while preserving competition excitement.

\begin{figure}[htbp]
    \centering
    \includegraphics[width=0.55\textwidth]{../figures/pipeline.png}
    \caption{Modeling Pipeline Overview} 
    \label{fig:modeling_pipeline}
\end{figure}


