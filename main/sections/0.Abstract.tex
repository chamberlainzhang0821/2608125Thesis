\begin{abstract}
Televised dance competitions combine observed judges' scores with undisclosed fan votes, making the elimination process only partially observable. Using multi-season data from \emph{Dancing with the Stars} (DWTS), we develop a unified modeling pipeline to reconstruct latent fan vote shares, quantify uncertainty, compare alternative vote-combination rules, explain performance/popularity drivers, and propose a tunable new voting mechanism.

For Task 1, we infer week-level fan vote shares on the probability simplex via a symmetric Dirichlet prior and a Stable-ABC (Approximate Bayesian Computation) procedure that enforces the season-specific historical elimination rule. Posterior summaries yield vote-share means and uncertainties, with week-level diagnostics (acceptance rate and average posterior standard deviation) indicating inference difficulty. A posterior predictive check achieves perfect agreement with observed eliminations across 229 elimination weeks when posterior means are replayed through the historical rule.

For Task 2, we perform counterfactual replays by holding inferred fan shares fixed while switching the aggregation rule among \texttt{Rank}, \texttt{Percent}, and \texttt{Bottom2+JudgeSave}. We quantify consistency, flip events, and contestant-level survival-week shifts, showing that \texttt{Percent} provides the most stable and interpretable magnitude-based aggregation, while judge-save mechanisms can induce qualitatively larger outcome shifts even when \texttt{Rank} and \texttt{Percent} agree.

For Task 3, we fit ridge-regularized linear models with cross-validation to explain three targets: \texttt{WeeksSurvived}, \texttt{AvgJudgeTotal}, and \texttt{AvgFanShare}. Grouped coefficient magnitudes highlight that the professional partner is the strongest signal for judge outcomes and remains influential for fan support, with moderate contributions from industry and geographic attributes; sensitivity analyses confirm robustness to regularization and resampling.

For Task 4, we propose \texttt{FairVote}, a two-parameter rule that explicitly trades off historical consistency, ``close-call'' excitement rate, and fan-protection behavior. A grid backtest with bootstrap confidence intervals and leave-one-season-out checks identifies an operating point that yields moderate consistency while maintaining a controlled intervention rate, enabling stakeholders to choose policy priorities transparently. Overall, our framework provides an auditable approach for reconstructing hidden preferences, diagnosing rule sensitivity, and supporting evidence-based voting-rule design.



\end{abstract}
